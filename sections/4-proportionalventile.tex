\section{Proportionalventile}
\subsection{Wegeschaltventil}

\graphiccol{wegeschaltventil1}
\graphiccol{wegeschaltventil2}
\begin{itemize}
\item Druck verursacht keine Kräfte in Bewegungsrichtung des Schiebers.
\item Betätigungskraft $F$ öffnet den Steuerschieber.
\item Volumenstrom $Q$ entsteht durch Druckdifferenz $P_A - P_B$.
\item Lechanschluss $P_L$ ist drucklos. 
\end{itemize}



\subsection{Proportionalwegeventil}
\graphiccol{proportionalwegeventil}
\begin{itemize}
\item Federn zentrieren den Steuerschieber.
\item Leckage an Ringspalt und Steuerkante.
\item Schieberposition $s$ \emph{stufenlos} und \emph{proportional} zur Betätigungskraft $F$.
\item Volumenstrom $Q$ proportional zum Ansteuerungssignal zur Betätigungskraft $F$. 
\end{itemize}



\subsection{4/3 Proportionalwegeventil}
\graphiccol{proportionalwegeventil2}
\begin{itemize}
\item Eine Steuerkante je Arbeitsanschluss $A$ und $B$. 
\item Je nach Richtung der Steuerung fliesst Volumenstrom von $A$ nach $B$ oder umgekehrt. 
\end{itemize}


\vfill
\newpage
