\section{Grundlagen}
\subsection{Analogien}
\begin{tabular}{ll}
Potentialgrösse & Druck $[bar]$ \\
Flussgrösse & Volumenstrom $[l/min]$ \\
Leistung & Druck $\cdot$ Volumenstrom $\frac{[bar] \cdot [l/min]}{0.6}$ \\
Widerstand & Druck/Volumenstrom $\frac{[bar]}{[l/min]}$ \\
Kapazität & Volumen/Druck $\frac{[m^3]}{[bar]}$ \\
Induktivität & Druck/Volumenstromänd. $\frac{[bar]}{[(l/min)/s}$
\end{tabular}
\subsection{Umrechungen}
Umrechnung von Kubikmeter zu Liter:
\begin{align*}
1\, m^3 = 1000 \, l
\end{align*}

\subsection{Kraft durch Druck}
Zusammenhang zw. Kraft und Fläche:
\begin{align*}
F = p \cdot A \tag{Kraft = Druck*Fläche}
\end{align*}

\subsection{Volumenstrom}
\begin{align*}
\dot{Q} = A \cdot v	
\end{align*}

\subsection{Kontinuitätsgleichung}
\begin{align*}
Q &= v \cdot A \tag{Massenerhalt} \\
v &= \frac{Q_1}{A_1} = \frac{Q_2}{A_2}
\end{align*}

\subsection{Leistung}
Mechanische Leistung:
\begin{align*}
P_M = F \cdot v
\end{align*}

Hydraulische Leistung:
\begin{align*}
P_H = Q \cdot p
\end{align*}


%\subsection{Hydrostatik / Hydrodynamik}
%Die \emph{Hydrostatik} nutzt die \emph{potentielle Energie} des Druckes in einem geschlossenen Volumen zur Leistungsübertragung. \\
%
%Die \emph{Hydrodynamik} nutzt die \emph{kinetische Energie} eines Fluids zur Leistungsübertragung. 




\subsection{Pascalsches Gesetz}
Gilt für zwei direkt verbundene Kolben mit unterschiedlichen Kolbenflächen $A_1$ und $A_2$ und unterschiedlichen Kräften $F_1$ und $F_2$. 
\myeqstar{
\frac{F_1}{F_2} = \frac{x_2}{x_1} = \frac{v_2}{v_1} = \frac{A_1}{A_2}
}
\begin{tabular}{llll}
$\rho:$ & Druck $[bar]$ & $F:$ & Kraft $[N]$ \\
$V:$ & Volumen $[m^3]$ & $x:$ & Weg $[m]$ \\
$Q:$ & Volumenstrom $[l/min]$ & $v:$ & Geschw. $[m/s]$\\
$A:$ & Fläche $[m^2]$
\end{tabular}

\vfill
\columnbreak
\subsection{Viskosität}
Die Viskosität ist sozusagen der Widerstand eines Fluids und daher stark von der Temperatur abhängig.

\graphicsm{viskositaet}

Die viskosität ist ein Mass für die Zähigkeit und bestimmt die notwendige Kraft $F$, um eine Platte mit der Kontaktfläche $A$ im Abstand $h$ mit der Relativgeschwindigkeit $v$ über einen Flüssigkeitsfilm zu ziehen. Sie stellt somit eine Grösse für die inneren Reibungskräfte dar. 

\myeqstar{
\frac{F}{A} &= \eta \cdot \frac{dv}{dz} \tag{Newton} \\
\eta &= \frac{F \cdot h}{A \cdot v_{(z=h)}} \tag{Dyn. Viskosität} \\
\nu &= \frac{\eta}{\rho} \tag{Kinem. Viskosität}
}
\begin{tabular}{llll}
$F:$ & Zugkraft $[N]$ & $\nu:$ & Kin. Viskosität $[m^2/s]$ \\
$v:$ & Relativgeschw. $[m/s]$ & $\eta:$ & Dyn. Viskosität $[Pa\,s]$ \\
$h:$ & Filmdicke $[m]$ & $\rho:$ & Dichte $[kg/m^2]$\\
$A:$ & Kontaktfläche $[m^2]$
\end{tabular}


\subsection{Viskosität \& Wirkungsgrad}
\graphiccol{visk_wirkungsgrad}
\vfill


\subsection{Viskosität beeinflussen}
Den grössten Einfluss auf die Viskosität hat die Temperatur. So wird auch je nach Anweundungsbereich der Temperatur entsprechend das Arbeitsfluid gewählt. 
\graphiccol{visk_temp}
Man beachte die zweifach logarithmische Darstellung. Der Bezugswert der kinematischen Viskosität wird bei $40^\circ C$ ermittelt.

Der Druck hat ebenfalls Auswirkungen auf die Viskosität, der allerdings bei Werten unter $100 bar$ vernachlässigt werden kann. Mit steigendem Druck nimmt auch die Viskosität zu. 
\vfill

\subsection{Dichte von Druckflüssigkeiten}
\graphiccol{dichte}

\myeqstar{
\frac{\Delta \rho}{\rho} &= \frac{1}{E} \cdot \Delta p \tag{T=const.} \\
\frac{\Delta \rho}{\rho} &= -\beta_T \cdot \Delta T \tag{p=const.} \\
E &= f(p) \tag{Kompressionsmodul} \\
\beta_T &= 6.5 -  7.5 \cdot 10^{-4} [K^{-1}] \tag{Ausdehnungs Koeff.}
}


\vfill
\columnbreak
\subsection{Elastizität der Druckflüssigkeit}
Eine Druckflüssigkeit arbeitet ähnlich einer mechanischer Feder, mit der Ausnahme dass keine Zugkräfte übertragen werden können. Massgebend ist der Kompressionsmodul $E$. Der Kompressionsmodul wird stark durch den Anteil freier Luftbläschen im Fluid beeinflusst. 

\myeqstar{
C(x) = \frac{E \cdot A}{x} \hspace{0.5cm} \left[\frac{N}{mm}\right] 
\tag{Federsteifigkeit der Druckflüssigkeit}
}
Vorsicht: $C$ steigt linear mit sinkender Höhe $x$ an, ist also nicht konstant!

\begin{tabular}{ll}
$E:$ & E-Modul Öl \\
$A:$ & Fläche Ölsäule \\
$x:$ & Höhe Ölsäule
\end{tabular} \\


Parallelschaltung von $N$ Federn:
\begin{align*}
C = \sum C_1  + C_2 + \hdots + C_N
\end{align*}


Serieschaltung von $N$ Federn:
\begin{align*}
\frac{1}{C} = \sum \frac{1}{C_1} + \frac{1}{C_2} + \hdots + \frac{1}{C_N}
\end{align*}


Federkraft:
\begin{align*}
F = C \cdot \Delta x
\end{align*}




\subsubsection{Normierter Druck}
Zusammenhang zwischen Weg- und Druckänderung der Ölsäule:
\myeqstar{
&\frac{\Delta p}{E} = \frac{\frac{\Delta x}{x_0}}{1- \frac{\Delta x}{x_0}} = \frac{\Delta x}{x_0 - \Delta x}\\
&\Delta x  = \left( \frac{P}{E + P} \right) x_0
}

Daraus folgt: 
$$\Delta x = \frac{x_0}{\frac{E}{\Delta p} + 1}$$

\subsubsection{Kompressionsvolumen}
Volumen, das aufgrund der Kompressibilität einer Flüssigkeit zusätzlich in einen Raum gedrückt werden muß, um dort eine bestimmte Druckänderung $\Delta p$ zu erzeugen:

\myeqstar{
\Delta V = \frac{V_0}{E} \cdot \Delta p \tag{Kompressionsvolumen} \\
\Delta p = \frac{\Delta V}{V} \cdot E \tag{Kompressionsgleichung}
}
\begin{tabular}{ll}
$V_0:$ & Volumen unter Druck \\
$\Delta p:$ &Druckerhöhung im Raum
\end{tabular}

\subsubsection{Druckerhöhung durch mechanische Kompression}
Kompression wird durch Wegänderung $\Delta x$ eines beweglichen Kolbens aufgeprägt.
\myeqstar{
\Delta V &= \Delta x \cdot A \tag{Komprimiertes Volumen} \\
V &= x \cdot A \tag{Volumen nach Kompr.} \\
\Delta p &= \frac{\Delta x}{x} \cdot E \tag{aus Komressionsgleichung} \\
\Delta p &= \frac{\Delta F }{A} \tag{Druckänderung aus Kraftglgw.}
}








\subsection{Druckänderung in offenen Systemen}
Druckänderungsgeschwindigkeit:
\myeqstar{
\dot{p} &= \frac{\Sigma Q_i}{V} \cdot E  \\
\dot{p} &= \left( \frac{\dot{m}_1 - \dot{m}_2}{m} \pm \frac{\dot{x}}{x} \right) \cdot E 
}



\vfill
\columnbreak
\subsection{Strömungen}
Für jede Geometrie ergeben sich andere Gleichungen für eine Strömung. Der Volumenstrom ist aber immer proportional zur Druckdifferenz und umgekehrt proportional zur dynamischen Viskosität. Da die Viskosität stark temperaturabhängig ist, hänt somit auch der Volumenstrom von der Temperatur ab. 

\subsubsection{Reynoldszahl}
Die Reynoldszahl gibt Auskunft darüber ob die Strömung laminar oder Turbulent ist. In Rohrströmungen sind ab einem Wert von $2315$ turbulente Strömungen möglich
\myeqstar{
Re &= \frac{\rho \cdot v_m \cdot d}{\eta} = \frac{v_m \cdot d}{\nu} = \frac{4}{\pi} \cdot \frac{Q}{\nu \cdot d} \\
d &= \frac{4 \cdot A}{U}
}
\begin{tabular}{llll}
$\eta:$ & Dyn. Viskosität $[Pa \, s]$ \qquad ($\eta = \nu \cdot \rho$)\\
$\nu:$ & Kin. Viskosität $[m^2/s]$ \qquad (aus Diagram) \\
$\rho:$ & Dichte $[kg/m^3]$  \\
$v_m:$ & Mittlere Geschw. $[m/s]$ \\
$d:$ & Hydr. Durchmesser $[m]$ \\
$Q:$ & Volumenstrom $[m^3/s]$
\end{tabular} \\


\subsubsection{Bernoulli Gleichung}
Bernnoulli Druckgleichung für reibungsfreie, inkompressible Medien:
\myeqstar{
p + \rho g z + \frac{\rho}{2} v^2 = const.
}

\begin{description}
\item[Statischer Druck:] Ergibt sich aus Druckmessung senkrecht zur Strömungsrichtung. Die kinetische Energie des Stromes hat keinen Einfluss, da sie nur in Strömungsrichtung wirksam ist.
\item[dynamischer Druck:] Druckmessung in Strömungsrichtung berücksichtigt auch die kinetische Energie eines Fluids. 
\end{description}





\subsubsection{Strömung im Rechteckspalt}
\graphiccol{rechteckspalt}
Kann benutzt werden um einen Leckage Volumenstrom zu berechnen, sofern die Voraussetzung erfüllt ist. Ein Rechteckspalt liegt vor, wenn die Höhe $h$ um mindestens eine Grössenordnung kleiner ist als die Breite $b$ und die Länge $l$ nochmals entsprechend grösser ist. \\


Voraussetzung: $h \ll b \ll l$ \\


\begin{align*}
v(x) = \frac{\Delta p}{2 \cdot \eta \cdot l} \cdot ( \frac{h^2}{4} - x^2)
\end{align*}

\myeqstar{
Q = \frac{\Delta p}{12 \cdot \eta \cdot l} \cdot b \cdot h^3 \tag{Volumenstrom}
}


\subsubsection{Strömung im Kreisspalt}
\graphiccol{kreisspalt}
Voraussetzung ist, dass die Läänge $l$ wesentlich grösser ist als der Durchmesser $2R$.
 
\begin{align*}
v(x) = \frac{\Delta p}{4 \cdot \eta \cdot l} \cdot (R^2 - x^2)
\end{align*}

\myeqstar{
Q = \frac{\Delta p}{8 \cdot \eta \cdot l} \cdot \pi \cdot R^4 \tag{Volumenstrom}
}


\subsubsection{Strömung im Kreisringspalt}
\graphiccol{kreisringspalt}

Voraussetzung: $r \ll R \ll l$

\begin{align*}
v(x) = \frac{\Delta p}{2 \cdot \eta \cdot l} \cdot \frac{(R-r)^2}{4} - x^2)
\end{align*}

\myeqstar{
Q = \frac{\Delta p}{12 \cdot \eta \cdot l} \cdot \pi \cdot (R+r) \cdot (R-r)^3
}


\subsubsection{Blendenströmung}
\graphiccol{blendensroemung}


\myeqstar{
Q = \alpha_D \cdot A_B \cdot \sqrt{\frac{2 \Delta p}{\rho}} \tag{Bernoullische Blendengleichung}
}

Die Einflüsse unterschiedlicher Blendengeometrien und Verluste durch Verwirblungen werden durch den experimentell ermittelten Kennwert $\alpha_D$ berücksichtigt.


\subsubsection{Druckverluste \& Reibungswiderstand}
\begin{itemize}
\item Druckverlust durch geometrischen Widerstand
\begin{align*}
\Delta p = \zeta \cdot \frac{\rho}{2} \cdot v_0^2 \tag{Druckverlust}\\
m 0 \frac{A_B}{A_0} = \left( \frac{d}{D} \right)^2 \tag{Blenden Öffnungsverhältnis} \\
\alpha_D = \frac{1}{m \cdot \sqrt{\zeta}} \tag{Blendenbeiwert} \\
\zeta_D = \frac{1}{m^2 \cdot \alpha_D^2} \tag{Widerstandszahl}
\end{align*}
\item Druckverluste in geraden Rohrleitungen
\begin{align*}
\Delta p = \lambda \cdot \frac{L}{d} \cdot \frac{\rho}{2} \cdot v^2
\end{align*}
\begin{tabular}{ll}
$\rho$ & Dichte \\
$\lambda$ & Rohrreibungszahl (aus Diagram)\\
$v$ & Mittlere Strömungsgeschw.
\end{tabular}
\end{itemize}


\subsubsection{Rohrreibungszahl}
\begin{align*}
\lambda = x \cdot \left( \frac{1}{Re} \right)^n
\end{align*}

\begin{tabular}{lcc}
Typ & x & n \\
\hline 
isotherm, laminar & 64 & 1 \\
laminar + Wärmeverlust & 75 & 1 \\
turbulent & 0.3164 & 0.25
\end{tabular}


\graphiccol{Rohrreibungszahl}

\subsection{Drücke und Geschwindigkeiten}

\begin{tabular}{l|c}
    Leitung & Fliessgeschwindigkeit \\
    \hline
    Saugleitung & $0.5 - 0.8 m/s$ \\
    Vorgespannte Saugleitung & $< 1.5 m/s$ \\
    Rücklauf- \& Tankleitung & $2 - 4 m/s$ \\
    Druckleitung $<$ 50 bar & $3 - 4 m/s$ \\
    Druckleitung $<$ 100 bar & $4 - 5 m/s$ \\
    Druckleitung $<$ 200 bar & $5 - 6 m/s$ \\
    Druckleitung $>$ 200 bar & $6 - 12 m/s$
\end{tabular}