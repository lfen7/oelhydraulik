\section{Speicher}
Anwendungen:\begin{itemize}
\item Schnellen Ausgleich von Volumenstrom 
\item Schwingsungsreduktion
\item Enthaltene Energie steht auch bei Stromausfall zur Verfügung
\item Regeneration von Bremsenergie
\end{itemize}

Verschiedene Arten von Speicher:
\begin{itemize}
\item Blasenspeicher:
\begin{itemize}
\item Gummiblase im Speicher enthält das Kompressionsgas.
\item  Sollte immer aufrecht stehen (gleichmässige Belastung der Blase).
\item Hohe Dynamik
\end{itemize}
\item Kolbenspeicher
\begin{itemize}
\item Trennkolben aus festem Material (Metall), besser geeignet für grössere Volumina ($>50 l$). 
\end{itemize}
\item Membranspeicher
\begin{itemize}
\item Dünne Membran trennt Medien.
\item Nur für sehr kleine Volumina geeignet ($<10 l$). 
\item Ausrichtung im Gegensatz zu Blasenspeicher egal. 
\end{itemize}
\end{itemize}


\subsection{Auslegung von Speichern}
\subsubsection{Volumenbedarf}
$0 \rightarrow 1$: Isotherm \\
 $1 \rightarrow 2$: Polytrop
\begin{align*}
V_0 = \frac{\Delta V}{\frac{p_0}{p_1} \cdot \left[ 1 - \left( \frac{p_1}{p_2} \right) ^{\frac{1}{n}} \right]}
\end{align*}

$0 \rightarrow 1$: Polytrop \\
 $1 \rightarrow 2$: Polytrop
\begin{align*}
V_0 = \frac{\Delta V}{\left( \frac{p_0}{p_1} \right)^{\frac{1}{n}} - \left( \frac{p_0}{p_2} \right)^{\frac{1}{n}}}
\end{align*}

\begin{tabular}{ll}
$n:$ & Polytropenexponent
\end{tabular} \\



\subsubsection{Pulsationsdämpfung}
\begin{align*}
\delta &= \frac{p_{Amplitude}}{p_{Mittelwert}} = \frac{p_2 - p_1}{p_2 + p_1} \\
V_0 &= \frac{p_1}{p_0} \cdot \frac{\Delta V}{1 - \left( \frac{1-\delta}{1+\delta}\right)^{\frac{1}{n}}}
\end{align*}


\subsection{Speicherdynamik}
\begin{align*}
\omega_s &= \sqrt{\frac{1}{C_H \cdot L_H}} \tag{Eigenfrequenz} \\
C_H &= \frac{V_m}{\kappa \cdot p_m} \tag{Speicherkapazität} \\
L_S &= \frac{m_k}{A_k^2} \tag{Speicherinduktivität} \\
L_A &= \frac{\rho \cdot l_S}{A_S} \tag{Anschlussinduktivität} \\
L_H &= L_S + L_A
\end{align*}

\begin{tabular}{ll}
$p_m:$ & Mittlerer Druck \\
$V_m:$ & Mittleres Volumen \\
$\kappa:$ & Isentropenexponent \\
$\rho:$ & Dichte des Öls \\
$A_K:$ & Querschnittfläche Trennelement \\
$m_k:$ & Masse Trennelement \\
$A_S:$ & Querschnittfläche Anschluss \\
$l_S:$ & Länge des Anschlusses
\end{tabular}



\vfill

