\section{Hydraulikzylinder}
Zylinder wandeln hydraulische Leistung mittels linearen Bewegungen in Kraft um.  


\subsection{Grundbauarten}
\subsubsection*{Gleichganzylinder}
\graphicsm{zyl_gleichgang}
Dank der durchgehenden Kolbenstange steht für beide Richtungen die gleiche Fläche zur Verfügung. Die Dynamik wird dadurch stark erhöht.


\subsubsection*{Differentialzylinder}
\graphicsm{zyl_differential}
Zylinder mit nur einer einseitigen Kolbenstange. Geeignet wenn nur in eine Richtung Arbeit verrichtet werden soll. Querkräfte werden jedoch schlechter aufgenommen.

Einziger Zylindertyp mit unterschiedlicher Verfahrgeschwindigkeit.


\subsubsection*{Plunger Zylinder}
\graphicsm{zyl_plunger}
Verfügen über nur einen Druckanschluss und brauchen daher einen weiteren Mechanismus um den Zylinder wieder einzufahren (externe Last oder Feder). 



\subsection{Grundgleichungen}
\graphiccol{zylinderkraefte}
\begin{align*}
&F_B = F_P - F_R - F_L = m \cdot a \\
&F_P - F_R = (p_A \cdot A_K - p_B \cdot A_R) \cdot \eta_{hm} \\
&v = \frac{Q_K}{A_K} = \frac{Q_R}{A_R} \tag{stationäre Geschw.}
\end{align*}
\begin{tabular}{ll}
$\eta_{hm}:$ & Hydraulisch-mechanischer Wirkungsgrad
\end{tabular} \\

\vfill
\columnbreak

\subsection{Knicksicherheit}
\subsubsection*{Eulerfaktor}
\graphiccol{eulerfall}

\subsubsection*{Korrekturfaktor}
\graphiccol{korrekturfaktor}

\subsubsection*{Schlankheitsgrad}
\myeqstar{
\lambda = 4 \cdot K \cdot \frac{h}{d_S}
}
\begin{tabular}{ll}
$K:$ & Korrekturfaktor \\
$h:$ & Hub \\
$d_S:$ & Durchmesser Stange
\end{tabular}


\subsubsection*{Grenzwert Knicksicherheit}
\myeqstar{
&\lambda_G = \pi \cdot \sqrt{\frac{E}{R_p}} \tag{Grenzwert} \\
\Rightarrow & \left( \frac{h}{d_S}\right)_G = \frac{\pi}{4 K} \cdot \sqrt{ \frac{E}{R_p}}
}

\begin{itemize}
\item Fall $\frac{h}{d_S} >= \left( \frac{h}{d_S}\right)_G; \quad \lambda \geq \lambda_G$:
Elastischer Biegefall nach Euler
\begin{align*}
\sigma_K = \frac{\pi^2 \cdot E}{\lambda^2}
\end{align*}
%
\item Fall $\frac{h}{d_S} < \left( \frac{h}{d_S}\right)_G; \quad \lambda \leq \lambda_G$: 
Plastischer Biegefall nach Tetmajer
\begin{align*}
\sigma_K = a-b \cdot \lambda
\end{align*}
\end{itemize}

Werte für $a$ und $b$: \\

\begin{tabular}{lcc}
Material & $a$ & $b$ \\
\hline \\
Baustahl E 335 & $335$ & $0.62$ \\
Vergütungsstahl 16NiCr4 & $470$ & $2.30$
\end{tabular}


\subsubsection*{Auslegung im Eulerfall für Differenzial-Zylinder}
\begin{align*}
F_{max} &= \frac{\pi^3}{64} \cdot \frac{d_S^4 \cdot E}{K^2 \cdot h^2} \\
h_{max} &= \frac{\pi \cdot d_S^2}{8 \cdot K} \cdot \sqrt{\frac{\pi \cdot E}{F_{max}} } \\
d_{S_{min}} &= \frac{4 d_K}{\pi d_S} \cdot K \cdot h \cdot \sqrt{\frac{\eta_{hm} \cdot p_{max}}{E} } \\
h_{max} &= \frac{\pi \cdot d_s^2}{4 K d_K} \cdot \sqrt{\frac{E}{\eta_{hm} \cdot p_{max}} }
\end{align*}


\newpage
