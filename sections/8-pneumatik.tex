\section{Pneumatik}
\subsection{Grundlagen}
\subsubsection{Ideales Gasgesetz}
\begin{align*}
p\cdot V  &= n \cdot R \cdot T \\
p \cdot V &= m \cdot R_s \cdot T
\end{align*}
\begin{tabular}{ll}
$p:$ & Druck \\
$V:$ & Volumen \\
$n:$ & Stoffmenge \\
$R:$ & Universelle Gaskonstante $[\frac{J}{mol \, K}]$ \\
$R_s:$ & Spezifische Gaskonstante (Luft: $R_s = 286.95 [\frac{J}{kg \, K}]$) \\
$T:$ & Temperatur
\end{tabular} \\

\begin{itemize}
\item Isothermer Prozess
\begin{align*}
p \cdot V = const.
\end{align*}
\item Isobarer Prozess
\begin{align*}
\frac{T}{V} = const. 
\end{align*}
\item Isochorer Prozess
\begin{align*}
\frac{T}{p} = const.
\end{align*}
\item Isentroper Prozess ($S=const.$)
\begin{align*}
&p\cdot V^\kappa = const.
&n=  \kappa = \frac{c_p}{c_v}
\end{align*}
\end{itemize}

\subsubsection{Spezifische Gaskonstante}
\begin{align*}
R = \frac{R_M}{M}
\end{align*}
\begin{tabular}{ll}
$R_M = 8.31:$ & Molare Gaskonstante $[\frac{J}{mol K}]$ \\
$M:$ & Molare Masse $[kg/mol]$
\end{tabular} \\


\subsubsection{Zustandsänderung idealer Gase}
\begin{align*}
\frac{p \cdot V}{T} &= const. \\
p \cdot V^n &= const.
\end{align*}
\begin{tabular}{ll}
$n:$ & Polytropenexponent
\end{tabular}


\subsection{Feuchte Luft}
\myeqstar{
p = p_L + p_W \tag{Gesamtluftdruck}
}
\begin{align*}
&p_w \cdot V = m_w \cdot R_w \cdot T  \\
&p_L \cdot V = m_L \cdot R_L \cdot T  \\
\end{align*}
\myeqstar{
x = \frac{m_w}{m_L} = \frac{R_L}{R_w} \cdot \frac{p_w}{p-p_w} \tag{Wassergehalt}
}

\begin{tabular}{ll}
$R_L:$ & 286.95 [J/KgK] \\
$R_W:$ & 461.50 [J/KgK] 
\end{tabular}


\subsubsection{Relative Luftfeuchtigkeit}
\graphiccol{taupunktkurve}
Unterhalb der Kurve bildet sich kein Kondensat, die relative Luftfeuchtigkeit ist $ \varphi < 100 \%$. 
Auf der Taupunktkurve ist  $\phi = 100 \%$. Kondensat bildet sich im Bereich oberhalb der Kurve. 

\myeqstar{
\varphi = \frac{m_{WR}}{m_{WRS}} \tag{Relative Luftfeuchtigkeit}
}
\begin{align*}
m_{WRS} = \frac{m_{WS}}{V} = \frac{p_{WS}}{R_W \cdot T}
\end{align*}

\begin{tabular}{ll}
$p_{wS}:$ & max. Partialdruck $[N/m^2]$ \\
$m_{WRS}:$ & Sättigung Relative Wassermenge $[kg]$ \\
$m_{WR}:$ & Relative Wassermenge $[kg/m^3]$
\end{tabular}

\subsection{Bauelemente der Pneumatik}
\subsubsection{Pneumatischermuskell}
\graphiccol{Pneumatischermuskel}
\begin{itemize}
\item kein Stick-Slip bis zum Stillstand
\item keine Dichtungen
\item geringer Energieverbrauch
\item Höchstens 1/3 des Querschnitts
\item Erheblich geringeres Gewicht
\item Billiger und einfacher in der Handhabung
\item positionieren in jeder Zwischenstellung
\end{itemize}

\graphiccol{Pneumatischermuskel_Kennlinienfeld}
\begin{itemize}
\item Materialdehnung (0 bar Kurve)
\item relative Verformung bezogen auf Baulänge [\%]
\item weitgehend lineare Kennlinie durch geringe Volumenänderung
\end{itemize}

\subsubsection{Vakumduese mit Saugnapfl}
\graphiccol{Vakumduese_mit_Saugnapf}
\begin{itemize}
\item Vakuumdüse arbeitet mit Venturiprinzip
\item Unterdruck am Saugnapf
\item Tank wird gefüllt
\item Tank entleert sich über Saugnapf bei Druckabschaltung
\item Luftimpuls stösst das gehaltene Werkstück ab
\end{itemize}
