\section{Hydraulische Widerstandssteuerung}
\graphicsm{signalumsetzung}
\myeqstar{
y_L = y_Q + y_p = \frac{Q_L}{C_0} + \frac{p_L}{E_0} = \frac{\Delta V \cdot A}{C_0} + \frac{\Delta F}{A \cdot E_0}
}

\begin{tabular}{ll}
$y_L:$ & Ventilöffnung \\
$\Delta v:$ & Geschwindigkeitsänderung \\
$A:$ & Wirkfläche \\
$E_0:$ & Druckverstärkung \\
$C_0:$ & Volumenstromverstärkung \\
$\Delta F:$ & Kraftänderung
\end{tabular} \\

Erhöhung von $C_0, E_0$ bewirkt:
\begin{itemize}
\item Kleinere Steuersignale nötig 
\item Höhere Laststeifigkeit 
\item Bessere Dynamik des Ventils
\end{itemize}


\subsection{Ventile - Grundlagen}
\myeqstar{
Y = \alpha_D \cdot \pi \cdot d \cdot \sqrt{\frac{2}{\rho}} \tag{Ventilkonstante}
}
\begin{tabular}{ll}
$Y:$ & Regelblende \\
$s:$ & Stellweg \\
$\alpha_D:$ & Blendenbeiwert
\end{tabular} \\


Ventilhub: 
\begin{itemize}
\item Negative Überdeckung (ständige Leckage):
\begin{align*}
y_N = s_N + s_0 
\end{align*}
\item Nullschnitt:
\begin{align*}
y_N = s_N
\end{align*}
\item Posiive Überdeckung (Längerer Stellweg)
\begin{align*}
y_N = s_N - s_0
\end{align*}
\end{itemize}


\subsection{Einkantensteuerung}
\graphiccol{einkantensteuerung}

\myeqstar{
Q_A = Y \cdot (s_0 + s) \cdot \sqrt{p_0 - p_A} - Y_T \cdot \sqrt{p_A} \tag{Lastvolumenstrom} 
}

\begin{tabular}{ll}
$Q_A:$ & Volumenstrom Verbraucher \\
$Y, Y_T:$ & Blendenkonstanten \\
$s:$ & Stellweg \\
$y_L:$  & Ventilöffnung $y_L = s + s_0$
\end{tabular} \\




%\myeqstar{
%\frac{p_A}{p_0} &= \frac{\left(1+\frac{s}{s_0} \right)^2}{\left( \frac{Y_T}{Y \cdot s_0} \right)^2 + \left(1 + \frac{s}{s_0} \right) ^2} = \frac{\left(1+\frac{s}{s_0} \right)^2}{ K^2 + \left(1 + \frac{s}{s_0} \right) ^2} \\
%Y_T &= \alpha_D \cdot A_B \cdot \sqrt{\frac{2}{\rho}} \tag{Festblende} \\
%Y &= \alpha_D \cdot \pi \cdot d \cdot \sqrt{\frac{2}{\rho}} \tag{Regelblende}
%}
%\begin{tabular}{ll}
%$Y:$ & Ventilkonstante \\
%$s:$ & Stellweg \\
%$\alpha_D:$ & Blendenbeiwert
%\end{tabular} \\

\myeqstar{
K = \frac{A_B}{\pi \cdot d \cdot s_0} \tag{Ventilkonstante}
}
\begin{tabular}{ll}
$A_B:$ & Fläche Verbraucheranschluss (Anschluss $A$)  \\
$d:$ & Durchmesser von Ventil Zylinder
\end{tabular} \\

Die Ventilkonstante ist das Flächenverhältnis der Blenden bei $s=0$, also in Mittelstellung. 

\myeqstar{
E_0(K=1) = \frac{1}{2}
 \cdot \frac{p_0}{s_0} \tag{Druckverstärkung}
}

\myeqstar{
C_0 = \frac{\Delta Q_A}{\Delta s} = \sqrt{\frac{1}{2}} \cdot Y \cdot \sqrt{p_0} \tag{Volumenstromverstärkung}
}

\myeqstar{
y_L = \sqrt{2} \cdot \frac{v \cdot A}{Y \cdot \sqrt{p_0}} + 2 \frac{F \cdot s_0}{A \cdot p_0}
}


\subsection{Zweikantensteuerung}
\graphiccol{zweikantensteuerung}

\myeqstar{
Q_A &= Y \cdot (s_0 + s) \cdot \sqrt{p_0 - p_A} - Y \cdot (s_0 - s) \cdot \sqrt{p_A} \tag{Lastvolumenstrom} \\
\frac{p_A}{p_0} &= \frac{1}{2} \cdot \frac{\left(1+\frac{s}{s_0} \right)^2}{1+\left( \frac{s}{s_0} \right)^2} \\
\\
E_0 &= \frac{p_0}{s_0} \tag{Druckverstärkung} \\
\\
C_0 &= \frac{\Delta Q_A}{\Delta s} = \sqrt{2} \cdot Y \cdot \sqrt{p_0} \tag{Volumenstromverstärkung} \\
y_L &= \frac{1}{\sqrt{2}} \cdot \frac{v \cdot A}{Y \cdot \sqrt{p_0}} +  \frac{F \cdot s_0}{A \cdot p_0}
}

%\begin{tabular}{ll}
%$Y:$ & Blendenkonstante 
%\end{tabular} \\

%\myeqstar{
%\frac{p_A}{p_0} = \frac{1}{2} \cdot \frac{\left(1+\frac{s}{s_0} \right)^2}{1+\left( \frac{s}{s_0} \right)^2}
%}
%
%\myeqstar{
%E_0 &= \frac{p_0}{s_0} \tag{Druckverstärkung} \\
%C_0 &= \frac{\Delta Q_A}{\Delta s} = \sqrt{2} \cdot Y \cdot \sqrt{p_0} \tag{Volumenstromverstärkung}
%}
%
%\myeqstar{
%y_L = \frac{1}{\sqrt{2}} \cdot \frac{v \cdot A}{Y \cdot \sqrt{p_0}} +  \frac{F \cdot s_0}{A \cdot p_0}
%}


\subsection{Vierkantensteuerung}
\graphiccol{vierkantensteuerung}
\myeqstar{
\frac{p_L}{p_0} &= \frac{2 \frac{s}{s_0}}{1+\left( \frac{s}{s_0} \right)^2 } \\
E_0 &= 2 \cdot \frac{p_0}{s_0}  \\
c_0 &= \sqrt{2} \cdot Y \cdot \sqrt{p_0}  \\
y_L &= \frac{1}{\sqrt{2}} \cdot \frac{v \cdot A}{Y \cdot \sqrt{p_0}} + \frac{1}{2} \frac{F \cdot s_0}{A \cdot p_0}
}
