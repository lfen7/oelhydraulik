\section{Servozylinder Antrieb}
\myeqstar{
Q_L &= Q_0 \cdot \frac{y}{y_{max}} \cdot \sqrt{1-\frac{p_L}{p_0}} \tag{Lastvolumenstrom} \\
Q_0 &= Q_N \cdot \sqrt{ \frac{p_0}{p_N} } \tag{Maximaler Ventilvolumenstrom}
}



\subsection{Kennlinienfeld}
\graphiccol{kennlinienfeld}
\begin{itemize}
\item Arbeitspunkt
\begin{align*}
\frac{p_L}{p_0}, \quad \frac{y}{y_{max}}
\end{align*}
\item Druckflussverstärkung:
\begin{align*}
V_{QY} = 
\frac{\partial Q_L}{\partial y} = \frac{Q_0}{y_{max}} \sqrt{1-\frac{p_L}{p_0}}
\end{align*}
\item Hydraulischer Leitwert:
\begin{align*}
V_{Qp} = \frac{\partial Q_L}{\partial p_L} = \frac{1}{2} \frac{Q_0}{p_0} \frac{\frac{y}{y_{max}}}{\sqrt{1-\frac{p_l}{p_0}}}
\end{align*}
\item Druckversärkung:
\begin{align*}
V_{py} = \frac{\partial p_L}{\partial y} = 2 \frac{p_0}{y} \cdot \left(1-\frac{p_L}{p_0} \right)
\end{align*}
\end{itemize}



\subsection{Dynamische Kennwerte}
Die Dynamischen Kennwerte beziehen sich auf die linearisierte Steuerkette.

\graphiccol{differentialgleichungssystem}

\graphiccol{steuerkette}
\begin{itemize}
\item Steuerverstärkung:
\begin{align*}
K = \frac{V_{QY}}{A}
\end{align*}
\item Dämpfungsgrad:
\begin{align*}
D &= \frac{1}{2 A} \left[ (V_{QP} + K_L ) \cdot \sqrt{\frac{2 m E}{V_0}} + R \cdot \sqrt{\frac{V_0}{2 m E}} \right] \\
\\
&\text{mit}\quad V_0 = 0.5 \,  h \, A: \\
D &= \frac{1}{2 A} \cdot \left[ (V_{Qp} + K_L) \cdot \sqrt{\frac{4  m \cdot E}{h \cdot A}} + R \sqrt{ \frac{h \cdot A}{4  m \cdot E}} \right]
\end{align*}
\begin{tabular}{ll}
$K_L:$ & Leckage Faktor \\
$R:$ & Reibungs-Faktor
\end{tabular} \\

\item Eigenfrequenz:
\begin{align*}
\omega_0 &= A \sqrt{\frac{2 E}{m V_0}} \\
\\
\text{mit} & \quad V_0 = 0.5 \, h \, A: \\
\omega_0 &= \sqrt{ \frac{4 \cdot E \cdot A}{m \cdot h} } 
\end{align*}
\item Maximale Kreisverstärkung mit einfachem P-Regler:
\begin{align*}
V_K = 2 \cdot D \cdot \omega_0 
\end{align*}
\end{itemize}


\subsubsection{Mit Zustandsregler}
Die veränderte Dynamik mit Zustandsregler ist
\begin{align*}
\bar{\omega}_0 &= \omega_0 \cdot \sqrt{1 + K \cdot K_v} \\
\\
\bar{D} &= \frac{D + 0.5\cdot K \cdot \omega_0 \cdot K_A}{\sqrt{1+K\cdot K_V}}
\end{align*}

\begin{tabular}{ll}
$K_V:$ & Geschwindigkeitsfaktor \\
$K_A:$ & Beschleunigungsfaktor
\end{tabular}\\

Für Überschwingungsfreises Positionieren:
\begin{align*}
K_P = 0.27 \cdot \bar{\omega}_0 \qquad \bar{D}_{opt} = 0.613
\end{align*}
